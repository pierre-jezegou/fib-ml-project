\documentclass[10pt,a4paper,hidelinks]{article}
\usepackage[utf8]{inputenc}
\usepackage[english]{babel}
\usepackage[T1]{fontenc}
\usepackage{lmodern}
\usepackage{xcolor}
\usepackage[left=1.5cm,
            right=1.5cm,
            top=0.75cm,
            bottom=1.75cm]{geometry}
% \usepackage[left=1cm, right=1cm, top=1cm, bottom=1cm]{geometry}

\setlength{\parindent}{0pt}
\usepackage[colorlinks = true,
            linkcolor = blue,
            urlcolor  = blue,
            citecolor = blue,
            anchorcolor = blue]{hyperref}

\usepackage{titling}
\newcommand{\subtitle}[1]{
    \posttitle{
    \par\end{center}
    \begin{center}\Large\bfseries#1\end{center}
    }
}

\setlength{\droptitle}{-3em}
\pretitle{\begin{center}\huge\bfseries} % Police et taille du titre
\posttitle{\end{center}} % Espacement après le titre
\preauthor{\begin{center}\large} % Police et taille des auteurs
\postauthor{\end{center}\vspace{-1em}} % Espacement après les auteurs
\predate{\begin{center}\large} % Police et taille de la date
\postdate{\end{center}\vspace{-2em}} % Espacement après la date

\renewcommand*\familydefault{\sfdefault}

\title{Project proposal}
\subtitle{Machine Learning Analysis of Train stations}
\date{}
\author{Pierre Jézégou \& Julie Oppedal}

\begin{document}
\maketitle

\section{Introduction, context \& motivations}
Analyzing the strategic placement of railway stations holds significant importance for transportation authorities and urban planners alike. It enables them to optimize infrastructure development, enhance accessibility, and promote sustainable urban growth. Leveraging advanced machine learning techniques and comprehensive datasets, this project aims to investigate the optimal meshing of railway stations in France, particularly focusing on the feasibility and impact of establishing new stations in urban areas.\\
Furthermore, through the application of predictive modeling methodologies learned in coursework, such as regression and classification, we aim to develop models capable of forecasting future demand patterns and assessing the economic viability of proposed station locations. By grounding our analysis in real-world data and industry practices, we aim to deliver actionable recommendations to policymakers and transportation stakeholders, facilitating informed decision-making and the sustainable development of railway networks across France.
\\

\textbf{How can machine learning techniques be effectively employed to analyze the optimal placement of railway stations in urban areas using readily available datasets, aiming to enhance accessibility, promote sustainable urban development, and inform decision-making processes for transportation authorities and urban planners?}

\section{Data}
\subsection{Data source \& Build our own dataset}
A wealth of data is available on the "SNCF" and "INSEE" websites: the company and the organization make available a vast amount of open data \url{https://ressources.data.sncf.com}, \url{https://statistiques-locales.insee.fr/}. It will therefore be possible to create our own dataset to study the features we find most interesting. We'll build the data by cross-referencing several databases, possibly enriched by external data (such as services provided, distance travelled, station traffic...). 
Construction of the dataset has already begun and is available on GitHub: \url{https://github.com/pierre-jezegou/fib-ml-project}.  This construction allows us to have enough data (especially features) to obtain an analysis as close as possible to reality.


Creating a dataset from these data doesn't seem impossible and isn't overly time-consuming. All the required data are already available online, requiring concatenation with identification keys such as station IDs or postal codes. The most time-consuming aspect will be preprocessing, which is part of the course pipeline.

\subsection{Data shape and characteristics}
The base dataset used before consolidation is this \href{https://ressources.data.sncf.com/explore/dataset/gares-de-voyageurs/}{link}. It has only \textit{2862 rows $\times$ 6 columns} before adding data from other databases to consolidate it. We cross-reference this table with \href{https://ressources.data.sncf.com/explore/dataset/gares-pianos/information/}{these data}, \href{https://ressources.data.sncf.com/explore/dataset/gares-equipees-du-wifi/information/}{this one}, and mainly \href{https://ressources.data.sncf.com/explore/dataset/frequentation-gares/information/}{these data}. We will then add some other data (computed or formatted) during preprocessing via calculation on existing features: travel distance...\\

Final dataset characteristics are \textbf{available in the \href{https://github.com/pierre-jezegou/fib-ml-project/blob/main/README.md}{README.md} file} of the project repository\\


In our dataset containing train stations and their characteristics, the elements can be considered relatively independent. For instance, individual features of a station such as its geographical location, capacity, or number of platforms can be seen as independent from each other. The data can be enriched as and when required by the many open data sources offered by \href{https://ressources.data.sncf.com/}{SNCF} and \href{https://statistiques-locales.insee.fr/}{INSEE}.

Consolidated data of train stations are available in the assets of the last \href{https://github.com/pierre-jezegou/fib-ml-project/releases/}{release}.\\

We will also be able to add demographic and financial data for train station cities, for more relevant features. This will require work on the data to enable cross-referencing.

\subsection{Studies proposed}
\begin{description}
    \item[Regression:] Regression problems linked to this dataset
        \begin{itemize}
            \item Predict the number of passengers in a station
            % \item Predict the number of inhabitants in an urban area, justifying whether or not a station is needed (ROI pre-studies)
        \end{itemize}
    \item[Classification:] Classification problems linked to this data
        \begin{itemize}
            \item Determine if a train station should have a given service (piano, wifi...)
        \end{itemize}
\end{description}

\end{document}