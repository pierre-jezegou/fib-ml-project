\chapter{Conclusion}

\section{Scientific and Personal Reflections}
This project was a great learning experience in the field of machine learning. We explored various techniques and worked with real-world data, which was both challenging and rewarding. We compared and implemented different methods, and used advanced tools like GridSearchCV. Online resources, especially the scikit-learn documentation, were very helpful.

Understanding the dataset before starting was difficult. Our dataset had limitations that made it hard to apply everything we learned in lectures, showing the gap between theory and practice.

\section{Possible Extensions and Known Limitations}
For future work, we could try new models and add more data when it becomes available, possibly next year. We could also use Generative Adversarial Networks (GANs) to create synthetic data, which might improve our models.

\section{Challenges Faced}
Throughout the project, we faced several challenges:

\begin{itemize}
    \item There were a lot of features to handle, which was overwhelming.
    \item The models showed a lot of differences in prediction accuracy and feature importance, making it hard to choose the best one.
    \item The dataset was not evenly distributed across the country, with too much data from Paris train stations, causing bias.
    \item Working with real-world data is complex. We had to understand the data well without changing it too much to avoid influencing the models.
    \item Dealing with non-optimized, real data was a constant challenge that required ongoing problem-solving. This project showed us the importance of being flexible and persistent when working with complex data.
\end{itemize}

In summary, this project gave us valuable insights into the practical side of machine learning. It reinforced our theoretical knowledge and highlighted the complexities and challenges of working with real-world data.